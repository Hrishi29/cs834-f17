\documentclass[letterpaper,11pt]{article}
\usepackage{graphicx}
\usepackage{listings}
\usepackage[super]{nth}
\usepackage[hyphens]{url}
\usepackage{hyperref}
\usepackage{amsmath}
\usepackage[makeroom]{cancel}
\usepackage[table]{xcolor}
\usepackage{comment}
\usepackage[space]{grffile}
\usepackage{csvsimple}
\usepackage{longtable}
\usepackage{adjustbox}


\newcommand*{\srcPath}{../src}%

\lstset{
	basicstyle=\footnotesize,
	breaklines=true,
}

\begin{document}

\begin{titlepage}

\begin{center}

\Huge{Assignment 1}

\Large{CS 734:  Introduction to Information Retrieval}

\Large{Fall 2017}

\Large{Hrishikesh Gadkari}

\Large Finished on \today

\end{center}

\end{titlepage}

\newpage


% =================================
% First question
% =================================
\section*{1}

\subsection*{Question}

\begin{verbatim}
1.   Suppose that, in an effort to crawl web pages faster, you set up two crawling machines with different starting seed URLs. Is this an effective strategy for distributed crawling? Why or why not? .
\end{verbatim}

\clearpage
\subsection*{Answer}

I think this is not a effective strategy for distributed crawling. Here the  machines are unable to share information with each other. Although the two machines are seeded with different URLs, this will result in alot of duplicated effort, as the two machines would eventually crawl many of the same URLs. This strategy could be made more effective by sharing the URL request queue between the two machines, thereby eliminating the duplicated effort.

\clearpage

% =================================
% Second question
% =================================

\section*{2}

\subsection*{Question}

\begin{verbatim}
2.   How would you design a system to automatically enter data in to web forms in order to crawl deep Web pages? What measures would you use to make sure your crawler’s actions were not destructive(for instance, so that it doesn’t add random blog comments). .
\end{verbatim}

\subsection*{Answer}


The system would use two strategies. The first, called FTF (Filling Text Fields), is how to
fill the fields efficiently, specially the text fields, which do not have a set of predetermined
values. The second strategy, called ITP (Instance Template Pruning),
is how to select queries to submit to a particular form in order to retrieve
more data with fewer submissions. The strategy to minimize the set of queries,
i.e., the number of form submissions, involves pruning the set of all possible
queries. As each query is submitted, data extracted from the resulting page is
used to identify wasteful queries and prune them.
\clearpage

% =================================
% 3rd question
% =================================

\section*{3}

\subsection*{Question}

\begin{verbatim}
3. List five web services or sites that you use that appear to use search, not including web search engines. Describe the role of search for that service. Also describe whether the search is based on a database or grep style of matching, or if the search isusing some type of ranking.
\end{verbatim}

\subsection*{Answer}

1. Alexa : I use this website to montor the website traffic, statistics and analytics. It uses the grep style of matching.

2. Youtube: It is based on index-database style matching. YouTube is essentially a search engine for videos. Not surprisingly, it uses a sophisticated ranking algorithm to surface content to viewers. If you want to gain a following and rank your videos higher in YouTube search, uploading fresh content is extremely important.

3. Amazon: It is based on grep+ ranking style search. Amazon Searches on Amazon are not based solely on keywords. Amazon's search is more of a marketing tool; displaying more popular and successful products over those that are less, even if the search words used better match new product pages. If you plan on just creating a new product page for your items and expect sales, that is not going to happen unless you invest time in marketing your items. As your product generates more page views, sales and product reviews, your product will rise within the search results. Also, you should try and see the difference within the results when searching 'All' as compared to searching within the category. You will likely see your item ranked higher when doing 'in-category' searches.

4. https://www.worldcat.org/ : I use this website to find items in libraries near my loaction. It uses grep style matching. Anything that is a product or catalogue search is a grep-like search.  The thing about grep style search is that its not very efficient, its normally sequential and you have no progress idicator (generally speaking).

5. www.odu.edu. For on-site searches it uses grep style of matching .


\clearpage


\end{document}